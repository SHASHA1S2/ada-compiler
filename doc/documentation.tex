\documentclass{article}
\usepackage[cm]{fullpage}
\usepackage{biblatex}
\usepackage[normalem]{ulem}
\bibliography{ref.bib}

\title{
Ada Compiler \\
CS355:Compiler Design
}
\author{
Anshu Avinash\\
\texttt{anshuavi@iitk.ac.in}
\and
Pranjal Singh\\
\texttt{spranjal@iitk.ac.in}
\and
Atique Firoz\\
\texttt{atiquef@iitk.ac.in}
\and
Parth Tripathi\\
\texttt{partht@iitk.ac.in}
}

\begin{document}
\maketitle
\textbf{Ada} is a strongly typed, modular, object oriented, concurrent, readable and expressible high-level computer programming language \cite{AdaIC}. In this project we would be creating a compiler for a subset of Ada language. We would be implementing following features:
\begin{itemize}
	\item \uline{Type System}: Ada's type system is governed by four principles: Strong typing, Static typing, Abstraction and Name equivalence.
	\item \uline{Exception}: Ada has modules which raise an error when certain conditions are not satisfied and another module which does corresponding error-handling.
	\begin{itemize}
		\item \uline{Predefined}: They are included in \emph{Standard} package. Some of them are: \textbf{Constraint\_Error, Program\_Error, Storage\_Error, Tasking\_Error}
		\item \uline{Input$/$Output}: These exceptions raised by subprograms of the predefined package \emph{Ada.Text\_IO}. Some of them are \textbf{End\_Error, Data\_Error,Mode\_Error,Layout\_Error}
		\item \uline{Raising exceptions}: The \emph{raise} statement explicitly raises a specified exception.
	\end{itemize}
\end{itemize}
\printbibliography
\end{document}
